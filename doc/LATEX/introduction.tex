\textbf{Darbo aktualumas} Mobilieji telefonai per pastaruosius metus tapo populiarūs dėl savo didelio funkcionalumo, prieinamumo ir produktyvumo. Tačiau šios pažangios technologijos taip pat atnešė naujų iššūkių ir pavojų, ypač kriminalistikoje. Kriminalistinėje veikloje mobilieji įrenginiai tapo duomenų kaupimo ir komunikacijos priemonėmis, o juose sukaupti duomenys tapo svarbia medžiaga teisminėse ekspertizėse.\\

\textbf{Darbo problema} Duomenų tyrimai rankiniu būdu užima daug laiko bei yra mažiau tikslūs ir saugūs, nes daugumą procedūrų norint išgauti duomenis yra pasikartojančios ir galėtų būti automatizuotos naudojant skriptus, kurie automatiškai išgautų duomenis iš atvaizdo. Be to, pats atvaizdo sukūrimas sudaro galimybes efektyvesnei analizei, nes keli darbuotojai gali dirbti su tuo pačiu atvaizdu, pasidalinti darbais. Deja, dėl žymių skirtumų tarp skirtingų mobiliųjų telefonų modelių bei operacinės sistemos variacijų, labai sunku sukurti įrankį, kuris automatiškai tiktų kiekvienam telefonui, be jokių modifikacijų.
„Android“ operacinėje sistemoje su kiekvienu atnaujinimu, duomenų pasiekimo kelias gali būti ribojamas. Mobilieji telefonai su „Android“ operacine sistema vis dažniau vartojami, todėl dažnai tenka daryti jų teisminę ekspertizę. Sukurtas įrankis palengvintų darbuotojams duomenų išgavimo procesą. \\

\textbf{Darbo objektas.}  Prijungus telefoną, sukūrus jo disko atvaizdą, sumontavus (angl. mounted) jį bei išgavus ir apdorojus duomenis, tyrėjas galėtų atlikti daug kokybiškesnį bei greitesnį darbą, naudojant automatizuotą įrankį. Įrankio veikimo principas paprastas - kompiuteryje įrašytas įrankis prisijungia prie telefono, ištraukia atminties atvaizdą, pasiekia jame esančią informaciją, tvarkingai ją suformatuoja ir paruošia tolimesniam tyrimui. Įrankis turėtų  išgauti duomenis vientisa tvarką, nes bet koks vientisumo pažeidimas sunaikintų duomenų integralumą.\\

\textbf{Darbo tikslas.}  Darbo tikslas yra sukurti automatizuotą įrankį, kuris ištirtų „Android“ įrenginį ir palengvintų jo tyrimo procesą. Šis darbas aprašo, kaip galima atlikti mobilių telefonų, veikiančių su „Android“ operacine sistema, teisminę ekspertizę.\\

\textbf{Darbo uždaviniai: }
\begin{itemize}
 \setlength{\itemsep}{1pt}
  \setlength{\parskip}{0pt}
  \setlength{\parsep}{0pt}
    \item  Apžvelgti Android įrenginio architektūrą
    \item  Pademonstruoti veikimo modelį
    \item  Apžvelgti sukurto įrankio funkcionalumą
    \item  Aprašyti ateities planus tyrimui
\end{itemize}
