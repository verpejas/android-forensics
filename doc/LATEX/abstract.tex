Išmanieji telefonai tapo vienu iš dažniausiai naudojamų komunikacijos priemonių dėl jų geriausio ryšio, funkcionalumo ir produktyvumo. Nuolat tobulėjant išmaniųjų telefonų technologijoms, atsiranda naujų lygių pavojai. Didžioji rinkos dalis pasitiki „Android“ operacinės sistemos telefonais, kuri yra svarbi jėga konkurencinėje rinkoje, „Android“ ir „iOS“ duopolyje. „Android“ telefonai kaupia milžinišką duomenų kiekį, kurį galima saugoti tiek vietiniu, tiek nuotoliniu būdu, todėl teismo ekspertams jie suteikia patikimų duomenų, kurie yra labai svarbūs teismo ekspertizei. \\

Šiame darbe siekiama sukurti dinamišką įrankį, skirtą duomenų išgavimui, analizei ir parengimui teismo ekspertizei. Įrankis sukurtas atsižvelgiant į tai, jog dauguma esamų įrankių nėra suderinami su visais „Android“ mobiliaisiais įrenginiais. Įrankio programavimui buvo naudojama „Shell Script“ kalbą. Visos komandas buvo automatizuotis ir suskirstytos į skirtingus skriptus, kurie palengvina teismo ekspertizės analizę. Šis įrankis gali būti  paleistais „Linux“ ar „Windows“ (su „Windows Subsystem for Linux“) operacinėse sistemose. Norint išgauti duomenis, naudojama „ADB Shell“ komunikacija, naudojant root lygį, kuris pasiekiamas paleidus neoficialų atstatymo režimą. Tyrimo metu buvo atsižvelgta į skambučių žurnalo istoriją, SMS žinutes, naršyklės istoriją, nuotraukas ir kitus failus, kurie yra svarbūs teismo ekspertizei. Norint palengvinti tolimesnę duomenų analizę, jie yra suskirstomi į aplankus. \\

Darbo eigoje sukurta priemonė išgauna įrenginio duomenų particijos atvaizdą, jį išsaugo ir tinkamai suformatuoja išgautus duomenis taip, kad būtų patogu juos naudoti bylų sprendimui. Sukurtas įrankis veikia su dauguma „Samsung“ įrenginių, kurie neturi įjungto priverstinio šifravimo (angl. force encrypt) funkcijos. Jis taip pat gali būti pritaikytas duomenų ištraukimui iš kitų gamintojų įrenginių.